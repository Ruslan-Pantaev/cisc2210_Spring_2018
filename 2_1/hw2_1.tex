   
\documentclass[11pt]{article}
\renewcommand{\baselinestretch}{1.05}
\usepackage{amssymb,amsmath,amsfonts,multirow}
\topmargin0.0cm
\headheight0.0cm
\headsep0.0cm
\oddsidemargin0.0cm
\textheight23.0cm
\textwidth16.5cm
\footskip1.0cm

 \begin{document}

\title{CISC 2210 Discrete Structures - Noson S. Yanofsky}
\author{Student: Ruslan Pantaev}
\maketitle

% ref <https://reu.dimacs.rutgers.edu/Symbols.pdf>

\section*{2.1}
%
%
\subsection*{1.}
\begin{center}
Let $p,q \text{ and } r$ be the following propositions:\\
$p =$ "it is raining,"\\
$q =$ "the sun is shining,"\\
$r =$ "there are clouds in the sky."\\
\hfill \break
Translate the following into logical notation, using $p,q,r$, and logical connectives.
\end{center}

\subsection*{(a)}
\begin{center}
It is raining and the sun is shining.\\
\hfill \break
$p \wedge q$
\end{center}

\subsection*{(b)}
\begin{center}
If It is raining, then there are clouds in the sky.\\
\hfill \break
$p \rightarrow r$
\end{center}

\subsection*{(c)}
\begin{center}
If It is not raining, then the sun is not shining and there are clouds in the sky.\\
\hfill \break
$\neg p \rightarrow (\neg q \wedge r)$
\end{center}

\subsection*{(d)}
\begin{center}
The sun is shining if and only if it is not raining.\\
\hfill \break
$q \longleftrightarrow \neg p$
\end{center}

\subsection*{(e)}
\begin{center}
If there are no clouds in the sky, then the sun is shining.\\
\hfill \break
$\neg r \rightarrow q$
\end{center}
%
%
\subsection*{2.}

\subsection*{(d)}
\begin{center}

\end{center}

\subsection*{(e)}
\begin{center}

\end{center}
%
%
\subsection*{3.}

\subsection*{(a)}
\begin{center}
Give truth values of the propositions in parts (a) to (e) of Example 1:\\
\end{center}

\begin{flushleft}
(a) Julius Caesar was president of the United States: False\\
(b) 2 + 2 = 4: True\\
(c) 2 + 3 = 7: False\\
(d) The number 4 is positive and the number 3 is negative: False\\
(e) If a set has $n$ elements, then it has $2^{n}$ subsets: True\\
(Bonus)\\
(f) $2^{n} + n $ is a prime number for infinitely many $n$: don't know...\\
(g) Every even integer greater than 2 is the sum of two prime numbers: no one knows... see "Goldbach's conjecture"\\
\end{flushleft}

\subsection*{(b)}
\begin{center}
Do the same for parts (a) and (b) of Example 2:
\end{center}

\begin{flushleft}
(a) $x + y = y + x$ for all $x,y \in \mathbb{R}$: True commutative property\\
(b) $2^{n} = n^{2}$ for some $n \in \mathbb{N}$: True for $\{2,4\}$\\
\end{flushleft}
%
%
\subsection*{9.}

\subsection*{(a)}
\begin{center}
Show that $n = 3$ provides one possible counterexample to the assertion "$n^{3} < 3^{n} \forall n \in \mathbb{N}$":\\

\end{center}

\subsection*{(b)}
\begin{center}
Can you find any other counterexamples?\\

\end{center}





\end{document}

   
\documentclass[11pt]{article}
\renewcommand{\baselinestretch}{1.05}
\usepackage{amssymb, amsmath, amsfonts, multirow}
\topmargin0.0cm
\headheight0.0cm
\headsep0.0cm
\oddsidemargin0.0cm
\textheight23.0cm
\textwidth16.5cm
\footskip1.0cm

 \begin{document}

\title{CISC 2210 Discrete Structures - Noson S. Yanofsky}
\author{Student: Ruslan Pantaev}
\maketitle

% ref <https://reu.dimacs.rutgers.edu/Symbols.pdf>

\section*{2.1}
%
%
\subsection*{1.}
\begin{center}
Let $p,q \text{ and } r$ be the following propositions:\\
$p =$ "it is raining,"\\
$q =$ "the sun is shining,"\\
$r =$ "there are clouds in the sky."\\
\hfill \break
Translate the following into logical notation, using $p,q,r$, and logical connectives.
\end{center}

\subsection*{(a)}
\begin{center}
It is raining and the sun is shining.\\
\hfill \break
$p \wedge q$
\end{center}

\subsection*{(b)}
\begin{center}
If It is raining, then there are clouds in the sky.\\
\hfill \break
$p \rightarrow r$
\end{center}

\subsection*{(c)}
\begin{center}
If It is not raining, then the sun is not shining and there are clouds in the sky.\\
\hfill \break
$\neg p \rightarrow (\neg q \wedge r)$
\end{center}

\subsection*{(d)}
\begin{center}
The sun is shining if and only if it is not raining.\\
\hfill \break
$q \longleftrightarrow \neg p$
\end{center}

\subsection*{(e)}
\begin{center}
If there are no clouds in the sky, then the sun is shining.\\
\hfill \break
$\neg r \rightarrow q$
\end{center}
%
%
\subsection*{2.}
\begin{center}
Let $p,q, \text{ and } r$ be as in Exercise 1. Translate the following into English sentences:
\end{center}

\subsection*{(d)}
\begin{center}
$\neg(p \longleftrightarrow (q \vee r))$\\
It is not raining if and only if the sun is not shining or there are no clouds in the sky.
\end{center}

\subsection*{(e)}
\begin{center}
$\neg(p \vee q) \wedge r$\\
It is not raining or the sun is not shining, but there are clouds in the sky.
\end{center}
%
%
\subsection*{3.}

\subsection*{(a)}
\begin{center}
Give truth values of the propositions in parts (a) to (e) of Example 1:\\
\end{center}

\begin{flushleft}
(a) Julius Caesar was president of the United States: False\\
(b) 2 + 2 = 4: True\\
(c) 2 + 3 = 7: False\\
(d) The number 4 is positive and the number 3 is negative: False\\
(e) If a set has $n$ elements, then it has $2^{n}$ subsets: True\\
(Bonus)\\
(f) $2^{n} + n $ is a prime number for infinitely many $n$: don't know...\\
(g) Every even integer greater than 2 is the sum of two prime numbers: no one knows... see "Goldbach's conjecture"\\
\end{flushleft}

\subsection*{(b)}
\begin{center}
Do the same for parts (a) and (b) of Example 2:
\end{center}

\begin{flushleft}
(a) $x + y = y + x$ for all $x,y \in \mathbb{R}$: True commutative property\\
(b) $2^{n} = n^{2}$ for some $n \in \mathbb{N}$: True for $\{2,4\}$\\
\end{flushleft}
%
%
\hfill \break
\hfill \break
\hfill \break
\hfill \break
%
%
\begin{center}
\textbf{\textit{Note}}\\
\textit{\hspace{0mm}Converse		: flips}\\
\textit{\hspace{8mm}Inverse		: negates}\\
\textit{\hspace{11mm}Contrapositive	: flips and negates}
\end{center}
%
%
\subsection*{6.}
\begin{center}
Give the converses of the following propositions:
% see <http://staff.scem.uws.edu.au/cgi-bin/cgiwrap/zhuhan/dmath/dm_readall.cgi?page=5&part=2>
\end{center}

\subsection*{(b)}
\begin{center}
If I am smart, then I am rich.\\
\hfill \break
If I am rich, then I am smart.
\end{center}

\subsection*{(c)}
\begin{center}
If $x^{2} = x \text{, then } x = 0 \text{ or } x = 1$.\\
\hfill \break
If $x = 0 \text{ or } x = 1 \text{, then } x^{2} = x$.
\end{center}
%
%
\subsection*{7.}
\begin{center}
Give the contrapositives of the propositions in Exercise 6:
% see <http://staff.scem.uws.edu.au/cgi-bin/cgiwrap/zhuhan/dmath/dm_readall.cgi?page=5&part=2>
\end{center}

\subsection*{(b)}
\begin{center}
If I am smart, then I am rich.\\
\hfill \break
If I am not rich, then I am not smart.
\end{center}

\subsection*{(c)}
\begin{center}
If $x^{2} = x \text{, then } x = 0 \text{ or } x = 1$.\\
\hfill \break
If $x \neq 0 \text{ or } x \neq 1 \text{, then } x^{2} \neq x$.
\end{center}
%
%
\subsection*{9.}

\subsection*{(a)}
\begin{center}
Show that $n = 3$ provides one possible counterexample to the assertion "$n^{3} < 3^{n} \forall n \in \mathbb{N}$":\\
\hfill \break
$3^{3} = 27, 3^{3} = 27$,\\
thus $27 \nless 27$
\end{center}

\subsection*{(b)}
\begin{center}
Can you find any other counterexamples?\\
\hfill \break
I could not find any other counterexamples within the $\mathbb{N}$ domain.
\end{center}
%
%
\subsection*{11.}

\subsection*{(a)}
\begin{center}
Show that $x = -1$ is a counterexample to $(x+1)^{2} \geq x^{2} \forall x \in \mathbb{R}$:\\
\hfill \break
$(-1 + 1)^{2},  -1^{2}$\\
$0 \ngeq 1$
\end{center}

\subsection*{(b)}
\begin{center}
Find another counterexample:\\
\hfill \break
$x = -2$\\
$(-2 + 1)^{2},  -2^{2}$\\
$1 \ngeq 4$
\end{center}

\subsection*{(c)}
\begin{center}
Can a non-negative number serve as an example?\\
\hfill \break
No, because $\forall x \in \mathbb{N} \text{, } x+1 > x$
\end{center}
%
%
\subsection*{12.}
\begin{center}
Find counterexamples to the following assertions:
\end{center}

\subsection*{(a)}
\begin{center}
$2^{n} - 1$ is prime for every $n \geq 2$\\
\hfill \break
Consider $h(n) = |\{k \in \mathbb{N} : k|n\}|$,\\
if $h(n) = 2$, then $n$ is prime.\\
\hfill \break
let $n = 4$,\\
$2^{4} - 1 = 15$,\\
$h(15) = |1,3,5,15| = 4$,\\
thus $n = 4$ is a counterexample.\\
\hfill \break
let $n = 6$,\\
$2^{4} - 1 = 63$,\\
$h(63) = |1,3,7,9,21,63| = 6$,\\
thus $n = 6$ is another counterexample.
\end{center}
%
%


\end{document}

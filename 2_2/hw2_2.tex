   
\documentclass[11pt]{article}
\renewcommand{\baselinestretch}{1.05}
\usepackage{amssymb, amsmath, amsfonts, multirow}
\topmargin0.0cm
\headheight0.0cm
\headsep0.0cm
\oddsidemargin0.0cm
\textheight23.0cm
\textwidth16.5cm
\footskip1.0cm

\begin{document}

\title{CISC 2210 Discrete Structures - Noson S. Yanofsky}
\author{Student: Ruslan Pantaev}
\maketitle

\section*{2.2}
%
%
\subsection*{1.}
\begin{center}
Give the converse and contrapositive for each of the following propositions:
\end{center}

\subsection*{(a)}
\begin{center}
$p \rightarrow (q \wedge r)$\\
\hfill \break
converse: $(q \wedge r) \rightarrow p$\\
contrapositive: $\neg (q \wedge r) \rightarrow \neg p$
\end{center}

\subsection*{(b)}
\begin{center}
if $x + y = 1$, then $x^{2} + y^{2} \geq 1$\\
\hfill \break
converse: if $x^{2} + y^{2} \geq 1$, then $x + y  = 1$\\
contrapositive: if $x^{2} + y^{2} \ngeq 1$, then $x + y  \neq 1$
\end{center}

\subsection*{(c)}
\begin{center}
if $2 + 2 = 4$, then $3 + 3 = 8$\\
\hfill \break
converse: $if 3 + 3 = 8, then 2 + 2 = 4$\\
contrapositive: $if 3 + 3 \neq 8, then 2 + 2 \neq 4$
\end{center}
%
%
\subsection*{3.}
\begin{center}
Consider the following propositions:\\
\end{center}

\subsection*{(a)}
\begin{center}
Which proposition is the converse of $p \rightarrow q$?\\
\hfill \break
$q \rightarrow p$
\end{center}

\subsection*{(b)}
\begin{center}
Which proposition is the contrapositive of $p \rightarrow q$?\\
\hfill \break
$\neg q \rightarrow \neg p$
\end{center}

\subsection*{(c)}
\begin{center}
Which propositions are logically equivalent to $p \rightarrow q$?\\
\hfill \break
$\neg p \vee q$\\
$\neg q \rightarrow \neg p$\\
$\neg (p \wedge \neg q)$
\end{center}
%
%
\subsection*{9.}
\begin{center}
Construct the truth table for $[(p \vee q) \wedge r] \rightarrow (p \wedge \neg q)$\\
\hfill \break
\begin{tabular}{ |c|c|c|c|c|c|c|c|} 
\hline
$p$ & $q$ & $r$ & $[(p \vee q)$ & $\wedge r]$ & $\rightarrow$ & $(p \wedge$ & $\neg q)$\\
\hline
\multirow{1}{1em}
{ 0}	&	0	&	0	&	0	&	0	&	\textbf{1}	&	0	&	1\\
0	&	0	&	1	&	0	&	0	&	\textbf{1}	&	0	&	1\\
0	&	1	&	0	&	1	&	0	&	\textbf{1}	&	0	&	0\\
0	&	1	&	1	&	1	&	1	&	\textbf{0}	&	0	&	0\\
1	&	0	&	0	&	1	&	0	&	\textbf{1}	&	1	&	1\\
1	&	0	&	1	&	1	&	1	&	\textbf{1}	&	1	&	1\\
1	&	1	&	0	&	1	&	0	&	\textbf{1}	&	0	&	0\\
1	&	1	&	1	&	1	&	1	&	\textbf{0}	&	0	&	0\\
\hline
\end{tabular}
\end{center}
%
%
\subsection*{11.}
\begin{center}
Construct truth table for:
\end{center}

\subsection*{(a)}
\begin{center}
$\neg (p \vee q) \rightarrow r$\\
\hfill \break
\begin{tabular}{ |c|c|c|c|c|c|} 
\hline
$p$ & $q$ & $r$ & $\neg$ & $(p \vee q)$ & $\rightarrow r$\\
\hline
\multirow{1}{1em}
{ 0}	&	0	&	0	&	1	&	0	&	\textbf{0}\\
0	&	0	&	1	&	1	&	0	&	\textbf{1}\\
0	&	1	&	0	&	0	&	1	&	\textbf{1}\\
0	&	1	&	1	&	0	&	1	&	\textbf{1}\\
1	&	0	&	0	&	0	&	1	&	\textbf{1}\\
1	&	0	&	1	&	0	&	1	&	\textbf{1}\\
1	&	1	&	0	&	0	&	1	&	\textbf{1}\\
1	&	1	&	1	&	0	&	1	&	\textbf{1}\\
\hline
\end{tabular}
\end{center}
%
%
\hfill \break
\subsection*{12.}
\begin{center}

\end{center}

\subsection*{(a)}
\begin{center}

\hfill \break

\end{center}

\subsection*{(b)}
\begin{center}

\hfill \break

\end{center}

\subsection*{(c)}
\begin{center}

\hfill \break

\end{center}

\subsection*{(d)}
\begin{center}

\hfill \break

\end{center}

\subsection*{(e)}
\begin{center}

\hfill \break

\end{center}

\subsection*{(f)}
\begin{center}

\hfill \break

\end{center}

\subsection*{(g)}
\begin{center}

\hfill \break

\end{center}

\subsection*{(h)}
\begin{center}

\hfill \break

\end{center}
%
%



\end{document}

   
\documentclass[11pt]{article}
\renewcommand{\baselinestretch}{1.05}
\usepackage{amssymb, amsmath, amsfonts, multirow}
\topmargin0.0cm
\headheight0.0cm
\headsep0.0cm
\oddsidemargin0.0cm
\textheight23.0cm
\textwidth16.5cm
\footskip1.0cm

\begin{document}

\title{CISC 2210 Discrete Structures - Noson S. Yanofsky}
\author{Student: Ruslan Pantaev}
\maketitle

\section*{2.2}
%
%
\subsection*{1.}
\begin{center}
Give the converse and contrapositive for each of the following propositions:
\end{center}

\subsection*{(a)}
\begin{center}
$p \rightarrow (q \wedge r)$\\
\hfill \break
converse: $(q \wedge r) \rightarrow p$\\
contrapositive: $\neg (q \wedge r) \rightarrow \neg p$
\end{center}

\subsection*{(b)}
\begin{center}
if $x + y = 1$, then $x^{2} + y^{2} \geq 1$\\
\hfill \break
converse: if $x^{2} + y^{2} \geq 1$, then $x + y  = 1$\\
contrapositive: if $x^{2} + y^{2} \ngeq 1$, then $x + y  \neq 1$
\end{center}

\subsection*{(c)}
\begin{center}
if $2 + 2 = 4$, then $3 + 3 = 8$\\
\hfill \break
converse: $if 3 + 3 = 8, then 2 + 2 = 4$\\
contrapositive: $if 3 + 3 \neq 8, then 2 + 2 \neq 4$
\end{center}
%
%
\subsection*{3.}
\begin{center}
Consider the following propositions:\\
\end{center}

\subsection*{(a)}
\begin{center}
Which proposition is the converse of $p \rightarrow q$?\\
\hfill \break
$q \rightarrow p$
\end{center}

\subsection*{(b)}
\begin{center}
Which proposition is the contrapositive of $p \rightarrow q$?\\
\hfill \break
$\neg q \rightarrow \neg p$
\end{center}

\subsection*{(c)}
\begin{center}
Which propositions are logically equivalent to $p \rightarrow q$?\\
\hfill \break
$\neg p \vee q$\\
$\neg q \rightarrow \neg p$\\
$\neg (p \wedge \neg q)$
\end{center}
%
%
\subsection*{9.}
\begin{center}
Construct the truth table for $[(p \vee q) \wedge r] \rightarrow (p \wedge \neg q)$\\
\hfill \break
\begin{tabular}{ |c|c|c|c|c|c|c|c|} 
\hline
$p$ & $q$ & $r$ & $[(p \vee q)$ & $\wedge \hspace{1mm} r]$ & $\rightarrow$ & $(p \hspace{1mm} \wedge$ & $\neg q)$\\
\hline
\multirow{1}{1em}
{ 0}	&	0	&	0	&	0	&	0	&	\textbf{1}	&	0	&	1\\
0	&	0	&	1	&	0	&	0	&	\textbf{1}	&	0	&	1\\
0	&	1	&	0	&	1	&	0	&	\textbf{1}	&	0	&	0\\
0	&	1	&	1	&	1	&	1	&	\textbf{0}	&	0	&	0\\
1	&	0	&	0	&	1	&	0	&	\textbf{1}	&	1	&	1\\
1	&	0	&	1	&	1	&	1	&	\textbf{1}	&	1	&	1\\
1	&	1	&	0	&	1	&	0	&	\textbf{1}	&	0	&	0\\
1	&	1	&	1	&	1	&	1	&	\textbf{0}	&	0	&	0\\
\hline
\end{tabular}
\end{center}
%
%
\subsection*{11.}
\begin{center}
Construct truth tables for:
\end{center}

\subsection*{(a)}
\begin{center}
$\neg (p \vee q) \rightarrow r$\\
\hfill \break
\begin{tabular}{ |c|c|c|c|c|c|} 
\hline
$p$ & $q$ & $r$ & $\neg$ & $(p \vee q)$ & $\rightarrow r$\\
\hline
\multirow{1}{1em}
{ 0}	&	0	&	0	&	1	&	0	&	\textbf{0}\\
0	&	0	&	1	&	1	&	0	&	\textbf{1}\\
0	&	1	&	0	&	0	&	1	&	\textbf{1}\\
0	&	1	&	1	&	0	&	1	&	\textbf{1}\\
1	&	0	&	0	&	0	&	1	&	\textbf{1}\\
1	&	0	&	1	&	0	&	1	&	\textbf{1}\\
1	&	1	&	0	&	0	&	1	&	\textbf{1}\\
1	&	1	&	1	&	0	&	1	&	\textbf{1}\\
\hline
\end{tabular}
\end{center}
%
%
\hfill \break
\subsection*{12.}
\begin{center}
In which of the following statements is the "or" an "inclusive or"?
\end{center}

\subsection*{(a)}
\begin{center}
Choice of soup or salad - Exclusive Or
\end{center}

\subsection*{(b)}
\begin{center}
To enter the university, a student must have taken a year of chemistry or physics in high school - Inclusive Or
\end{center}

\subsection*{(c)}
\begin{center}
Publish or Perish - Exclusive Or
\end{center}

\subsection*{(d)}
\begin{center}
Experience with C++ of Java is desirable - Inclusive Or
\end{center}

\subsection*{(e)}
\begin{center}
The task will be completed on Thursday or Friday - Exclusive Or
\end{center}

\subsection*{(f)}
\begin{center}
Discounts are available to persons under 20 or over 60 - Exclusive Or
\end{center}

\subsection*{(g)}
\begin{center}
No fishing or hunting allowed - Inclusive Or
\end{center}

\subsection*{(h)}
\begin{center}
The school will not be open in July or August - Inclusive Or
\end{center}
%
%
\subsection*{13.}
\begin{center}
The exclusive or connective $\oplus$, is defined by the truth table:\\
\hfill \break
\begin{tabular}{ |c|c|c|} 
\hline
$p$ & $q$ & $p \oplus q$\\
\hline
\multirow{1}{1em}
{ 0}	&	0	&	\textbf{0}\\
0	&	1	&	\textbf{1}\\
1	&	0	&	\textbf{1}\\
1	&	1	&	\textbf{0}\\
\hline
\end{tabular}
\end{center}

\subsection*{(a)}
\begin{center}
Show that $p \oplus q$ has the same truth table as $\neg(p \leftrightarrow q)$:\\
\hfill \break
\begin{tabular}{ |c|c|c|c|} 
\hline
$p$ & $q$ & $\neg$ & $(p \leftrightarrow q)$\\
\hline
\multirow{1}{1em}
{ 0}	&	0	&	\textbf{0}	&	1\\
0	&	1	&	\textbf{1}	&	0\\
1	&	0	&	\textbf{1}	&	0\\
1	&	1	&	\textbf{0}	&	1\\
\hline
\end{tabular}
\end{center}

\subsection*{(b)}
\begin{center}
Construct a truth table for $p \oplus p$:\\
\hfill \break
This is a contradiction\\
\hfill \break
\begin{tabular}{ |c|c|c|} 
\hline
$p$ & $p$ & $p \oplus p$\\
\hline
\multirow{1}{1em}
{ 0}	&	0	&	\textbf{0}\\
0	&	0	&	\textbf{0}\\
1	&	1	&	\textbf{0}\\
1	&	1	&	\textbf{0}\\
\hline
\end{tabular}
\end{center}

\subsection*{(c)}
\begin{center}
Construct a truth table for $(p \oplus q) \oplus r$:\\
\hfill \break
\begin{tabular}{ |c|c|c|c|c|} 
\hline
$p$ & $q$ & $r$ & $(p \oplus q)$ & $\oplus \hspace{1mm} r$\\
\hline
\multirow{1}{1em}
{ 0}	&	0	&	0	&	0	&	\textbf{0}\\
0	&	0	&	1	&	0	&	\textbf{1}\\
0	&	1	&	0	&	1	&	\textbf{1}\\
0	&	1	&	1	&	1	&	\textbf{0}\\
1	&	0	&	0	&	1	&	\textbf{1}\\
1	&	0	&	1	&	1	&	\textbf{0}\\
1	&	1	&	0	&	0	&	\textbf{0}\\
1	&	1	&	1	&	0	&	\textbf{1}\\
\hline
\end{tabular}
\end{center}

\hfill \break
\hfill \break
\hfill \break
\hfill \break
\hfill \break
\hfill \break
\hfill \break
\hfill \break
\hfill \break
\subsection*{(d)}
\begin{center}
Construct a truth table for $(p \oplus p) \oplus p$:\\
\hfill \break
\begin{tabular}{ |c|c|c|c|c|} 
\hline
$p$ & $p$ & $p$ & $(p \oplus p)$ & $\oplus \hspace{1mm} p$\\
\hline
\multirow{1}{1em}
{ 0}	&	0	&	0	&	0	&	\textbf{0}\\
0	&	0	&	0	&	0	&	\textbf{0}\\
0	&	0	&	0	&	0	&	\textbf{0}\\
0	&	0	&	0	&	0	&	\textbf{0}\\
1	&	1	&	1	&	0	&	\textbf{1}\\
1	&	1	&	1	&	0	&	\textbf{1}\\
1	&	1	&	1	&	0	&	\textbf{1}\\
1	&	1	&	1	&	0	&	\textbf{1}\\
\hline
\end{tabular}
\end{center}
%
%
\subsection*{16.}

\subsection*{(a)}
\begin{center}
Write a compound proposition that is true when exactly one of the three propositions $p, q, \text{ and } r$ is true:\\
\hfill \break
$(p \vee q \vee r)$
\end{center}
%
%
\subsection*{18.}
\begin{center}
Prove or disprove:
\end{center}

\subsection*{(c)}
\begin{center}
$[(p \rightarrow q) \rightarrow r] \Longleftrightarrow [p \rightarrow (q \rightarrow r)]$\\
\hfill \break
This is true by way of the transitivity of $\rightarrow$ law.
\end{center}
%
%
\hfill \break
\hfill \break
\hfill \break
\hfill \break
\hfill \break
\hfill \break
\hfill \break
\hfill \break
\hfill \break
\hfill \break
\hfill \break
\hfill \break
\subsection*{19.}
\begin{center}
Verify the following logical equivalences using truth tables
\end{center}

\subsection*{(a)}
\begin{center}
Rule 12a: $[(p \rightarrow r) \wedge (q \rightarrow r)] \Longleftrightarrow [(p \vee q) \rightarrow r]$\\
\hfill \break
\begin{tabular}{ |c|c|c|c|c|c|c|c|c|} 
\hline
$p$ & $q$ & $r$ & $[(p \rightarrow r)$ & $\wedge$ & $(q \rightarrow r)]$ & $\leftrightarrow$ & $[(p \vee q)$ & $\rightarrow r]$\\
\hline
\multirow{1}{1em}
{ 0}	&	0	&	0	&	1	&	1	&	1	&	\textbf{1}	&	0	&	1\\
0	&	0	&	1	&	1	&	1	&	1	&	\textbf{1}	&	0	&	1\\
0	&	1	&	0	&	1	&	0	&	0	&	\textbf{1}	&	1	&	0\\
0	&	1	&	1	&	1	&	1	&	1	&	\textbf{1}	&	1	&	1\\
1	&	0	&	0	&	0	&	0	&	1	&	\textbf{1}	&	1	&	0\\
1	&	0	&	1	&	1	&	1	&	1	&	\textbf{1}	&	1	&	1\\
1	&	1	&	0	&	0	&	0	&	0	&	\textbf{1}	&	1	&	0\\
1	&	1	&	1	&	1	&	1	&	1	&	\textbf{1}	&	1	&	1\\
\hline
\end{tabular}
\end{center}
%
%
\subsection*{20.}
\begin{center}
Verify the following logical implications using truth tables
\end{center}

\subsection*{(b)}
\begin{center}
Disjunctive syllogism, rule 21: $[(p \vee q) \wedge \neg p] \Longrightarrow q$\\
\hfill \break
\begin{tabular}{ |c|c|c|c|c|c|} 
\hline
$p$ & $q$ & $[(p \vee q)$ & $\wedge$ & $\neg p]$ & $\rightarrow q$\\
\hline
\multirow{1}{1em}
{ 0}	&	0	&	0	&	0	&	1	&	\textbf{1}\\
0	&	1	&	1	&	1	&	1	&	\textbf{1}\\
1	&	0	&	1	&	0	&	0	&	\textbf{1}\\
1	&	1	&	1	&	0	&	0	&	\textbf{1}\\
\hline
\end{tabular}
\end{center}
%
%
\subsection*{22.}
\begin{center}
Prove or disprove the following. Don't forget that only \textit{one} line of the truth table is needed to show that a proposition is \textit{not} a tautology.
\end{center}

\subsection*{(a)}
\begin{center}
$(q \rightarrow p) \Longleftrightarrow (p \wedge q)$\\
\hfill \break
The following disproves that this is a tautology:\\
\hfill \break
\begin{tabular}{ |c|c|c|c|} 
\hline
$p$ & $q$ & $(q \rightarrow p)$ & $(p \wedge q)$\\
\hline
\multirow{1}{1em}
{ 0}	&	1	&	0	&	0\\
\hline
\end{tabular}
\end{center}
%
%
\hfill \break
\subsection*{23.}
\begin{center}
A logician told her son "If you don't finish your dinner, you will not get to stay up and watch TV."\\
He finished his dinner and then went straight to bed. Discuss:\\
\hfill \break
The logician's compound proposition spoke nothing of what happens if the son finishes his dinner, thus the son simply went to bed.\\
$p$ = finish dinner\\
$q$ = stay up\\
$r$ = watch tv\\
Logician's compound proposition: $\neg p \rightarrow \neg (q \wedge r)$\\
\hfill \break
As shown in this truth table, even when the son finished his dinner $(p = 1)$, the logical implication is no staying up and no TV! The son got the short end of the stick when it came to vacuous truths. This just goes to show that he should have simply not finished dinner \textit{and} stayed up late to watch TV; that would have been logically sound [GoTo row4].\\
\hfill \break
\begin{tabular}{ |c|c|c|c|c|c|c|} 
\hline
$p$ & $q$ & $r$ & $\neg p$ & $\rightarrow$ & $\neg$ & $(q \wedge r)$\\
\hline
\multirow{1}{1em}
{ 0}	&	0	&	0	&	1	&	\textbf{1}	&	1	&	0\\
0	&	0	&	1	&	1	&	\textbf{1}	&	1	&	0\\
0	&	1	&	0	&	1	&	\textbf{1}	&	1	&	0\\
\textit{0}	&	\textit{1}	&	\textit{1}	&	1	&	\textbf{0}	&	0	&	1\\
1	&	0	&	0	&	0	&	\textbf{1}	&	1	&	0\\
1	&	0	&	1	&	0	&	\textbf{1}	&	1	&	0\\
1	&	1	&	0	&	0	&	\textbf{1}	&	1	&	0\\
1	&	1	&	1	&	0	&	\textbf{1}	&	0	&	1\\
\hline
\end{tabular}
\end{center}
%
%
\subsection*{24.}
\begin{center}
Consider the statement "Concrete does not grow if you do not water it."\\
\hfill \break
Let $p$ = you water it\\
Let $q$ = concrete grows\\
\hfill \break
\begin{tabular}{ |c|c|c|} 
\hline
$\neg p$ & $\neg q$ & $\neg p \rightarrow \neg q$\\
\hline
\multirow{1}{1em}
{ 0}	&	0	&	\textbf{1}\\
0	&	1	&	\textbf{1}\\
1	&	0	&	\textbf{0}\\
1	&	1	&	\textbf{1}\\
\hline
\end{tabular}
\end{center}

\subsection*{(a)}
\begin{center}
Give the contrapositive:\\
\hfill \break
If concrete grows, then you water it.\\
\end{center}

\subsection*{(b)}
\begin{center}
Give the converse:\\
\hfill \break
If concrete does not grow, then you do not water it.
\end{center}

\subsection*{(c)}
\begin{center}
Give the converse of the contrapositive:\\
\hfill \break
If you water it, then concrete grows.
\end{center}

\subsection*{(d)}
\begin{center}
Which among the original statement and the ones in parts (a), (b), and (c) are true?\\
\hfill \break
\textit{I'm not a construction worker so I cannot vouch for the validity of any of these statements, haha}\\
\hfill \break
Original - True\\
(a) - False\\
(b) - True\\
(c) - False\\
\hfill \break
\textit{Really... who knows?...}
\end{center}

\end{document}

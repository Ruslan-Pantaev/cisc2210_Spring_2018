   
\documentclass[11pt]{article}
\renewcommand{\baselinestretch}{1.05}
\usepackage{amssymb, amsmath, amsfonts, multirow, tikz}
\usetikzlibrary{datavisualization}
\topmargin0.0cm
\headheight0.0cm
\headsep0.0cm
\oddsidemargin0.0cm
\textheight23.0cm
\textwidth16.5cm
\footskip1.0cm

 \begin{document}

\title{CISC 2210 Discrete Structures - Noson S. Yanofsky}
\author{Student: Ruslan Pantaev}
\maketitle

% ref <https://reu.dimacs.rutgers.edu/Symbols.pdf>

\section*{1.5}
%
%
\subsection*{1.}
\begin{center}
Let $f(n) = n^{2} + 3$ and $g(n) = 5n - 11$ for $n \in \mathbb{N}$\\
Thus $f: \mathbb{N} \rightarrow \mathbb{N}$ and $g: \mathbb{N} \rightarrow \mathbb{Z}$ Calculate:
\end{center}

\subsection*{(a)}
\begin{center}
$f(1)$ and $g(1)$\\
$f(1) = 1^{2} + 3 = 4$\\
$g(1) = 5(1) - 11 = -6$\\
\end{center}

\subsection*{(b)}
\begin{center}
$f(2)$ and $g(2)$\\
$f(2) = 2^{2} + 3 = 7$\\
$g(2) = 5(2) - 11 = -1$\\
\end{center}

\subsection*{(c)}
\begin{center}
$f(3)$ and $g(3)$\\
$f(3) = 3^{2} + 3 = 12$\\
$g(3) = 5(3) - 11 = 4$\\
\end{center}

\subsection*{(d)}
\begin{center}
$f(4)$ and $g(4)$\\
$f(4) = 4^{2} + 3 = 19$\\
$g(4) = 5(4) - 11 = 9$\\
\end{center}

\subsection*{(e)}
\begin{center}
$f(5)$ and $g(5)$\\
$f(5) = 5^{2} + 3 = 28$\\
$g(5) = 5(5) - 11 = 14$\\
\end{center}

\subsection*{(f)}
\begin{center}
To think about: Is $f(n) + g(n)$ always an even number?\\
\hfill \break
let's test base case of n = 0 to be sure:\\
$f(0)$ and $g(0)$\\
$f(0) = 0^{2} + 3 = 3$\\
$g(0) = 5(0) - 11 = -11$\\
\hfill \break
let $f(n)$ and $g(n)$ both produce either an even or odd integer within the $\mathbb{Z}$ domain.\\
for $n = \{n: n \in \mathbb{N}$ and $n \bmod 2 = 0\}$\\
let $z_{1}$ and $z_{2}$ be outputs of $f(n)$ and $g(n)$ respectively,\\
where $z = \{z: z \in \mathbb{Z}$ and $z \bmod 2 = 0\}$\\
thus, $(z_{1} + z_{2}) \bmod 2 = 0$\\
\hfill \break
let $z_{1}+1$ and $z_{2}+1$ be outputs of $f(n)$ and $g(n)$ respectively,\\
where $z = \{z: z \in \mathbb{Z}$ and $z \bmod 2 = 0\}$\\
thus, $((z_{1} + 1) + (z_{2} + 1)) \bmod 2 = (z_{1} + z_{2} + 2) \bmod 2 = 0$\\
\hfill \break
Yes, $f(n) + g(n)$ consistently produces an even integer within the $\mathbb{Z}$ domain.\\
\end{center}
%
%
\subsection*{2.}
\begin{center}
Consider the function $h: \mathbb{P} \rightarrow  \mathbb{P}$ defined by $h(n) = |\{k \in \mathbb{N} : k|n\}|$\\
for $n \in \mathbb{P}$. In words, $h(n)$ is the number of divisors of $n$.\\
Calculate $h(n)$ for $1 \leq n \leq 10$ and for $n = 73$.\\
\hfill \break
h(1) =	1\\
h(2) =	|1,2| = 2\\
h(3) =	|1,3| = 2\\
h(4) =	|1,2,4| = 3\\
h(5) =	|1,5| = 2\\
h(6) =	|1,2,3,6| = 4\\
h(7) =	|1,7| = 2\\
h(8) =	|1,2,4,8| = 4\\
h(9) =	|1,3,9| = 3\\
h(10) =	|1,2,5,10| = 4\\
h(73) =	|1,73| = 2\\
\end{center}
%
%
\hfill \break
\hfill \break
\hfill \break
\hfill \break
\hfill \break
\subsection*{3.}
\begin{center}
Let $\sum$* be the language using letters from $\sum = \{a,b\}$.\\
We've already seen a useful function from $\sum$* to $\mathbb{N}$.\\
It is the length function, which already has a name: length. Calculate:\\
\end{center}

\subsection*{(a)}
\begin{center}
length($bab$) = 3
\end{center}

\subsection*{(b)}
\begin{center}
length($aaaaaaaa$) = 8
\end{center}

\subsection*{(c)}
\begin{center}
length($\lambda$) = 0
\end{center}

\subsection*{(d)}
\begin{center}
What is the image set Im(length) for this function?\\
\hfill \break
$\mathbb{N}$
\end{center}
%
%
\subsection*{5.}
\begin{center}
Let $f$ be the function in example 3:\\
$f(m,n) = \lfloor\frac{n}{2}\rfloor - \lfloor\frac{m-1}{2}\rfloor$
\end{center}

\subsection*{(a)}
\begin{center}
Calculate $f(0,0), f(8,8), f(-8,-8), f(73,73),$ and $f(-73,-73)$\\
\hfill \break
$f(0,0) = \lfloor\frac{0}{2}\rfloor - \lfloor\frac{0-1}{2}\rfloor = 0 - (-1) = 1$\\
$f(8,8) = \lfloor\frac{8}{2}\rfloor - \lfloor\frac{8-1}{2}\rfloor = 4 - 3 = 1$\\
$f(-8,-8) = \lfloor\frac{-8}{2}\rfloor - \lfloor\frac{-8-1}{2}\rfloor = -4 - (-5) = 1$\\
$f(73,73) = \lfloor\frac{73}{2}\rfloor - \lfloor\frac{73-1}{2}\rfloor = 36 - 36 = 0$\\
$f(-73,-73) = \lfloor\frac{-73}{2}\rfloor - \lfloor\frac{-73-1}{2}\rfloor = -36 - (-36) = 0$\\
\end{center}

\subsection*{(b)}
\begin{center}
Find $f(n,n)$ for all $(n,n)$ in $\mathbb{Z} x \mathbb{Z}$.\\
\textit{Hint: Consider the cases when n is even and when it is odd}\\
\hfill \break
if $n \bmod 2 = 0$, $f(n,n) = 1$,\\
else if $n \bmod 2 = 1$, $f(n,n) = 0$
\end{center}
%
%
\subsection*{8.}
\begin{center}
Let $S = \{1,2,3,4,5\}$ and consider the functions $1_{s}. f. g \text{ and } h$ from $S \text{ into } S$\\
defined by $1_{s}(n) =n,f(n) = 6 -n, g(n) = max\{3,n\}, \text{ and } h(n) = max\{1, n-1\}$
\end{center}

\subsection*{(a)}
\begin{center}
Write each of these functions as a set of ordered pairs, i.e., list the elements in their graphs:\\
\hfill \break
$1_{s}(n) = \{(1,1), (2,2), (3,3), (4,4), (5,5)\}$\\
$f(n) = \{(1,5), (2,4), (3,3), (4,2), (5,1)\}$\\
$g(n) = \{(1,3), (2,3), (3,3), (4,4), (5,5)\}$\\
$h(n) = \{(1,1), (2,1), (3,2), (4,3), (5,4)\}$\\
\end{center}

\subsection*{(b)}
\begin{center}
Sketch a graph of each of these functions:\\
\hfill \break
\begin{tikzpicture}
\datavisualization [school book axes, visualize as smooth line]
	data {
		x, y
		1,1
		2,2
		3,3
		4,4
		5,5
	};
\end{tikzpicture}
%
\begin{tikzpicture}
\datavisualization [school book axes, visualize as smooth line]
	data {
		x, y
		1,5
		2,4
		3,3
		4,2
		5,1
	};
\end{tikzpicture}
%
\begin{tikzpicture}
\datavisualization [school book axes, visualize as smooth line]
	data {
		x, y
		1,3
		2,3
		3,3
		4,4
		5,5
	};
\end{tikzpicture}
%
\begin{tikzpicture}
\datavisualization [school book axes, visualize as smooth line]
	data {
		x, y
		1,1
		2,1
		3,2
		4,3
		5,4
	};
\end{tikzpicture}
\end{center}
%
%
\hfill \break
\hfill \break
\subsection*{9.}
\begin{center}
For $n \in \mathbb{Z}$, let $f(n) = \frac{1}{2}[(-1)^{n} + 1]$.\\
The function $f$ is the characteristic function for some subset of $\mathbb{Z}$.\\
Which subset?\\
\hfill \break
\{0,1\}, where 1 = all even integers in $\mathbb{Z}$
\end{center}
%
%
\hfill \break
\subsection*{10.}
\begin{center}
Consider subsets $A \text{ and } B$ of a set $S$
\end{center}

\subsection*{(a)}
\begin{center}
The function $X_{A} \cdot X_{B}$ is the characteristic function of some subset of $S$.\\
Which subset?\\
\hfill \break
$A \bigcap B$
\end{center}

\subsection*{(b)}
\begin{center}
Repeat part (a) for the function $X_{A} + X_{B} - X_{A \bigcap B}$\\
\hfill \break
$A \bigcup B$
\end{center}

\subsection*{(c)}
\begin{center}
Repeat part (a) for the function $X_{A} + X_{B} - 2 \cdot X_{A \bigcap B}$\\
\hfill \break
$A \backslash B \bigcup B \backslash A$
\end{center}
%
%
\subsection*{13.}
\begin{center}
We define functions mapping $\mathbb{R} \text{ into } \mathbb{R}$ as follows:\\
$f(x) = x^{3} - 4x$\\
$g(x) = \frac{1}{x^{2} + 1}$\\
$h(x) = x^{4}$
\end{center}

\subsection*{(a)}
\begin{center}
Find $f \circ f$\\
\hfill \break
$f(x^{3} - 4x)$\\
$= (x^{3} - 4x)^{3} - 4(x^{3} - 4x)$
\end{center}

\subsection*{(e)}
\begin{center}
Find $f \circ g \circ h$\\
\hfill \break
$f(g(h(x)))$\\
$= f(g(x^{4}))$\\
$= f(\frac{1}{(x^{4})^{2} + 1})$\\
$= (\frac{1}{(x^{4})^{2} + 1})^{3} - 4(\frac{1}{(x^{4})^{2} + 1})$
\end{center}

\end{document}
